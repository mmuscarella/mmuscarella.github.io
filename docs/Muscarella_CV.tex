%%%%%%%%%%%%%%%%%%%%%%%%%%%%%%%%%%%%%%%%%%%%%%%%%%%%%%%%%%%%%%%%%%%%%%%%%%%%%%%%
% Medium Length Professional CV
% LaTeX Template
% Version 2.0 (8/5/13)
%
% This template has been downloaded from:
% http://www.LaTeXTemplates.com
%
% Original author:
% Trey Hunner (http://www.treyhunner.com/)
%
% Important note:
% This template requires the resume.cls file to be in the same directory as the
% the .tex file. The resume.cls file provides the resume style used for
% structuring document.
%
%%%%%%%%%%%%%%%%%%%%%%%%%%%%%%%%%%%%%%%%%%%%%%%%%%%%%%%%%%%%%%%%%%%%%%%%%%%%%%%%

%-------------------------------------------------------------------------------
%	PACKAGES AND OTHER DOCUMENT CONFIGURATIONS
%-------------------------------------------------------------------------------

\documentclass{resume} % Use the custom resume.cls style

% Document margins
\usepackage[left=0.75in,top=0.75in,right=0.75in,bottom=0.75in]{geometry}

% Font
\usepackage[scaled]{helvet}
%\usepackage{mathptmx} % Times
\renewcommand\familydefault{\sfdefault}
\usepackage[T1]{fontenc}
\usepackage{hanging}


% \overfullrule=2cm


% Your name, address, phone number, and email
\name{Mario E. Muscarella}
\address{D\'{e}partement des Sciences Biologiques, Universit\'{e} du Queb\'{e}c \`{a} Montr\'{e}al}
\address{Phone: (438) 887-8669; E-mail: mmuscar@illinois.edu}
\address{Aquatic Ecology Research Group}

\begin{document}
\thispagestyle{empty}

%-------------------------------------------------------------------------------
%	EDUCATION SECTION
%-------------------------------------------------------------------------------

\begin{rSection}{Education}

  {\bf Indiana University} \hfill {\em Aug 2012 -- Jun 2016} \\
  Biology, Ph.D.; Evolution, Ecology \& Behavior Program

  {\bf Michigan State University} \hfill {\em June 2010 -- Aug 2012} \\
  Microbiology \& Molecular Genetics \\
  Ecology and Evolutionary Biology

  {\bf Armstrong State University} \hfill {\em Aug 2004 -- May 2008} \\
  Biology, Summa Cum Laude, B.S.

\end{rSection}

%-------------------------------------------------------------------------------
%	WORK EXPERIENCE SECTION
%-------------------------------------------------------------------------------

\begin{rSection}{Experience}

%--Appointment------------------------------------------------------------------
    \begin{rSubsection}{Postdoctoral Fellow}
      {July 2019 -- Present}{Universit\'{e} du Queb\'{e}c \`{a} Montr\'{e}al}{Montr\'{e}al, QC}
      \item del Giorgio Lab: Aquatic Ecology Research Group, Aquatic Microbial Ecology, Microbial Metabolism
    \end{rSubsection}

%--Appointment------------------------------------------------------------------
    \begin{rSubsection}{Postdoctoral Research Associate}
      {August 2016 -- June 2019}{University of Illinois}{Urbana, IL}
      \item O'Dwyer Lab: Theoretical Ecology, Microbial Ecology, Trait-Based Ecology, Phylogenetic Patterns
    \end{rSubsection}

%--Appointment------------------------------------------------------------------
    \begin{rSubsection}{Instructor}
      {August 2018 -- June 2019}{University of Illinois}{Urbana, IL}
      \item Courses Taught: Evolution of Molecules and Cells (Integrative Biology Honors Program)
    \end{rSubsection}

%--Appointment------------------------------------------------------------------
  \begin{rSubsection}{Graduate Research Assistant}
    {June 2010 -- July 2016}{Indiana University}{Bloomington, IN}
    \item Lennon Lab: Aquatic Microbial Ecology, Molecular and Ecological Strategies of Resource Utilization; \\
                      Interactions between Microbes and Molecules
  \end{rSubsection}

%--Appointment------------------------------------------------------------------
  \begin{rSubsection}{Associate Instructor}{January 2013 -- May 2015}
    {Indiana University}{Bloomington, IN}
    \item Courses Taught: Quantitative Biodiversity; Environmental Microbiology;
                          Introductory Biology Lab
  \end{rSubsection}

%--Appointment------------------------------------------------------------------
  \begin{rSubsection}{Graduate Teaching Assistant}{June 2011 -- August 2012}
    {Michigan State University}{East Lansing, MI}
    \item Courses Taught: Microbial Metagenomics; Cells and Molecules Lab;
                          Introductory Microbiology Lab
  \end{rSubsection}

%--Appointment------------------------------------------------------------------
  \begin{rSubsection}{Research Technician}{June 2008 -- June 2010}
    {University of Georgia Marine Institute}{Sapelo Island, GA}
    \item Booth Lab: Georgia Coastal Ecosystems LTER, \& Sapelo Island Microbial
                     Observatory
  \end{rSubsection}

%--Appointment------------------------------------------------------------------
  %\begin{rSubsection}{Undergraduate Research Assistant}{Sep 2005 -- May 2008}
  %  {Armstrong State University}{Savannah, GA}
  %  \item Awong-Taylor Lab: Environmental Microbiology
  %\end{rSubsection}

%--Appointment------------------------------------------------------------------
  %\begin{rSubsection}{Labaratory Technician}{Sep 2005 -- May 2008}
  %  {Armstrong State University}{Savannah, GA}
  %  \item Introductory Microbiology Labs: Media and Reagent Preperation;
  %                                        Lab Oversight
  %\end{rSubsection}

\end{rSection}

%-------------------------------------------------------------------------------
%	PUBLICATIONS SECTION
%-------------------------------------------------------------------------------

\begin{rSection}{Publications}

%--Publication------------------------------------------------------------------
{\bf Muscarella ME}, Lennon JT (In Review) Trait-based approach to bacterial growth efficiency. 
bioRxiv preprint available, DOI: https://doi.org/10.1101/427161

%--Publication------------------------------------------------------------------
{\bf Muscarella ME}, O'Dwyer JP (In Review) Species dynamics and interactions via metabolically informed consumer-resource models. 
bioRxiv preprint available, DOI: https://doi.org/10.1101/518449 

%--Publication------------------------------------------------------------------
{\bf Muscarella ME}, O'Dwyer JP (In Revision) Ecological insights from the evolutionary history of microbial innovations. 
bioRxiv preprint available, DOI: https://doi.org/10.1101/220939

%--Publication------------------------------------------------------------------
{\bf Muscarella ME}, Boot CM, Broeckling CD, Lennon JT (2019) Resource heterogeneity structures aquatic bacterial communities. 
The ISME Journal, Online Early: DOI: https://doi.org/10.1038/s41396-019-0427-7

%--Publication------------------------------------------------------------------
Lennon JT, {\bf Muscarella ME}, Placella SA, Lehmkuhl, BK (2018)
How, When, and Where Relic DNA Affects Microbial Diversity.
mBio, 9(3):e00637-18

%--Publication------------------------------------------------------------------
Aanderud ZT, Saurey S, Ball BA, Wall DH, Barrett JE, {\bf Muscarella ME}, Griffin NA, Virginia RA, Adams BJ (2018) 
Stoichiometric shifts in soil C:N:P promotes bacterial taxa dominance, maintains biodiversity, and deconstructs community assemblages. Frontiers in Microbiology, 9:1401

%--Publication------------------------------------------------------------------
Long H, Sung W, Kucukyildirim S, Williams E, Miller S, Guo W, Patterson C, Gregory C, Strauss C, Stone C, 
Berne C, Kysela D, Shoemaker WR, {\bf Muscarella ME}, Luo H, Lennon JT, Brun YV, Lynch M (2018) 
Evolutionary determinants of genome-wide nucleotide composition. Nature Ecology \& Evolution, 2(2):237-240

%--Publication------------------------------------------------------------------
Peralta AL, {\bf Muscarella ME}, Mathews JW (2017) 
Lingering land use legacies after different wetland restoration strategies.
Elementa: Science of the Anthropocene, 5:74

%--Publication------------------------------------------------------------------
Kuo V, Shoemaker WR, {\bf Muscarella ME}, Lennon JT (2017) 
Whole-genome sequence of the soil bacterium {\em Micrococcus} sp. KBS0714.
Genome announcements, 5(32):e00697-17

%--Publication------------------------------------------------------------------
Kelly PT, Bell T, Reisinger AJ, Spanbauer T, Bortolotti L, Brentrup J, Briseno-Avena C, Dong X,  Flanagan A, Follett E, Grosse J, Guy-Haim T, 
Holgerson MA, Hovel R, Luo J, Millette N, Mine A, {\bf Muscarella ME}, Oliver S, Smith H (2017) 
Ecological Dissertations in the Aquatic Sciences (Eco-DAS): An effective networking and professional development opportunity for early career aquatic scientists.
Limnology and Oceanography Bulletin, 26(2):25-30

%--Publication------------------------------------------------------------------
Guy-Haim T, Alexander H, Bell TW, Bier RL, Bortolotti LE, Briseno-Avena C,
Dong X, Flanagan AM, Grosse J, Grossmann L, Hasnain S, Hovel R, Johnston CA,
Miller DR, {\bf Muscarella ME}, Noto AE, Reisinger AJ, Smith HJ, Stamieszkin K
(2017) What are the type, direction, and strength of species, community, and
ecosystem responses to warming in aquatic mesocosm studies and their dependency
on experimental characteristics? A systematic review protocol.
Environmental Evidence, 6(1):6

%--Publication------------------------------------------------------------------
{\bf Muscarella ME}, Jones SE, Lennon JT (2016) Species sorting along a
subsidy gradient alters bacterial community stability. Ecology, 97(8):2034-2043

%--Publication------------------------------------------------------------------
Shoemaker WR*, {\bf Muscarella ME}*, Lennon JT (2015) Genome sequence of the
soil bacterium {\em Janthinobacterium} sp. KBS071. Genome Announcements,
3(3):e00689-15

%--Publication------------------------------------------------------------------
{\bf Muscarella ME}, Bird KC, Larsen ML, Placella SA, Lennon JT (2014)
Phosphorus resource heterogeneity affects the structure and function of
microbial food webs. Aquatic Microbial Ecology, 73(3):259-272

%--Publication------------------------------------------------------------------
Lennon JT, Hamilton SK, {\bf Muscarella ME}, Grandy AS, Wickings K, Jones SE
(2013) A source of terrestrial organic carbon to investigate the browning of
aquatic ecosystems. PLoS ONE, 8(10):e75771

%--Publication------------------------------------------------------------------
Awong-Taylor J, Craven K, Griffiths L, Bass C, {\bf Muscarella ME} (2008)
Comparison of biochemical and molecular methods for the identification of
bacterial isolates associated with failed loggerhead sea turtle eggs. Journal of
Applied Microbiology, 104:1244-1251

%--Publication------------------------------------------------------------------
Craven KS, Awong-Taylor J, Griffiths L, Bass C, {\bf Muscarella ME} (2007)
Identification of bacterial isolates from unhatched loggerhead
({\em Caretta caretta}) sea turtle eggs in Georgia, USA.
Marine Turtle Newsletter, 115:9-11

%--Note-------------------------------------------------------------------------
{\em (* co-first authors)}

\end{rSection}

\pagebreak

%-------------------------------------------------------------------------------
%	GRANTS SECTION
%-------------------------------------------------------------------------------

\begin{rSection}{Grants}

%--Grant------------------------------------------------------------------------
    \begin{rSubsection}{NSF Doctoral Dissertation Improvement Grant
      {\normalfont \em \$19,004}}{2015 -- 2017}{}{}
      \item DISSERTATION RESEARCH: Metabolic Resource Partitioning: Scaling
      Microbial Physiology from Individual Activity to Ecosystem Function
    \end{rSubsection}

%--Grant------------------------------------------------------------------------
    \begin{rSubsection}{Indiana Academy of Sciences Senior Research Grant
      {\normalfont \em \$2,200}}{2014 -- 2015}{}{}
      \item Metabolic Fate of Terrestrial Carbon Resources: Anabolic vs.
      Catabolic Processes
    \end{rSubsection}

%--Grant------------------------------------------------------------------------
    \begin{rSubsection}{Huron Mountain Wildlife Foundation Grant Renewal
      {\normalfont \em \$3,299}}{2012 -- 2014}{}{}
      \item Browning of Freshwater Ecosystems: Will Terrestrial Carbon Loading
      Alter the Diversity and Function of Aquatic Microbial Communities?
    \end{rSubsection}

%--Grant------------------------------------------------------------------------
    \begin{rSubsection}{Huron Mountain Wildlife Foundation Grant
      {\normalfont \em \$5,600}}{2011 -- 2012}{}{}
      \item Browning of Freshwater Ecosystems: Will Terrestrial Carbon Loading
      Alter the Diversity and Function of Aquatic Microbial Communities?
    \end{rSubsection}

\end{rSection}

%-------------------------------------------------------------------------------
%	HONORS AND AWARDS SECTION
%-------------------------------------------------------------------------------

\begin{rSection}{Honors and Awards}

%--Award------------------------------------------------------------------------
    \begin{rSubsection}{Postdoctoral Recruitment Fellowship}{2019 -- 2020}{}{}
      \item Universit\'{e} du Queb\'{e}c \`{a} Montr\'{e}al, Montr\'{e}al, QC
    \end{rSubsection}

%--Award------------------------------------------------------------------------
    \begin{rSubsection}{Teachers Ranked as Excellent}{2018}{}{}
      \item Integrative Biology, University of Illinois, Urbana, IL
    \end{rSubsection}

%--Award------------------------------------------------------------------------
    \begin{rSubsection}{Floyd Microbiology Summer Fellowship}{2013 -- 2014}{}{}
      \item Department of Biology, Indiana University, Bloomington, IN
    \end{rSubsection}

%--Award------------------------------------------------------------------------
    \begin{rSubsection}{Floyd Microbiology Travel Award}{2012}{}{}
      \item Department of Biology, Indiana University, Bloomington, IN
    \end{rSubsection}

%--Award------------------------------------------------------------------------
    \begin{rSubsection}{University Distinguished Fellowship}{2010}{}{}
      \item Graduate School, Michigan State University, East Lansing, MI
    \end{rSubsection}

%--Award------------------------------------------------------------------------
    \begin{rSubsection}{Early Start Fellowship}{2010}{}{}
      \item College of Natural Sciences, Michigan State University, East
      Lansing, MI
    \end{rSubsection}

%--Award------------------------------------------------------------------------
    \begin{rSubsection}{Les Davenport Award}{2008}{}{}
      \item Biology Department, Armstrong State University. Savannah, GA
    \end{rSubsection}

%--Award------------------------------------------------------------------------
    \begin{rSubsection}{ASB Research Award in Microbiology}{2008}{}{}
      \item Presented at the 69th Annual Association of Southeastern Biologists
      Conference \\
      Best Paper Presentation in the Area of Microbiology
    \end{rSubsection}

%--Award------------------------------------------------------------------------
    \begin{rSubsection}{Faculty Recognition Award}{2007}{}{}
      \item Biology Department, Armstrong State University. Savannah, GA
    \end{rSubsection}

\end{rSection}

%-------------------------------------------------------------------------------
%	SYMPOSIA SECTION
%-------------------------------------------------------------------------------

\begin{rSection}{Symposia and Working Groups}

  \begin{tabular}{ @{} >{\bfseries}l @{\hspace{6ex}} l }
  2018 & Microorganisms and Organic Carbon in the Marine Subsurface \\
       & Center for Dark Energy Biosphere Investigations, Knoxville, Tennessee \\
  2016 & Eco-DAS XII: Ecological Dissertations in the Aquatic Sciences, Honolulu, HI  \\
  \\
  2016 & Predicting the Response of Host-associated Microbiomes to Disturbance \\
       &  Santa Fe Institute Working Group, Santa Fe, NM \\
  2016 & Ramon Margalef Summer Colloquia \\
       & Microbes in a Changing World: Diversity and Biogeochemistry, Barcelona, Spain \\
  \end{tabular}

\end{rSection}

\pagebreak

%-------------------------------------------------------------------------------
%	WORKSHOPS SECTION
%-------------------------------------------------------------------------------

\begin{rSection}{Workshops}

  \begin{tabular}{ @{} >{\bfseries}l @{\hspace{6ex}} l }
  2018 & JGI Microbial Genomics and Metagenomics, DOE Joint Genome Institute, Walnut Creek, CA \\
  2013 & Software Carpentry Workshop, Bloomington IN \\
  2012 & DOM Fluorescence Workshop, INSTAAR, Boulder CO \\
  2010 & Mothur Software Workshop, Detroit MI \\
  \\
  \end{tabular}

\end{rSection}

%-------------------------------------------------------------------------------
%	SERVICE SECTION
%-------------------------------------------------------------------------------

\begin{rSection}{Service}

%--Service----------------------------------------------------------------------
    \begin{rSubsection}{Reviewer}{}{}{}
        \item \textbf{Journals}: {\em Aquatic Ecology, Aquatic Microbial Ecology, Environmental Microbiology, \\ 
        Environmental Science \& Technology, Limnology \& Oceanography Letters, Microbial Ecology, \\ 
        Microbiome Nature Communications, Proceedings of the Royal Society B, \\ 
        Science of the Total Environment, The ISME Journal}
        \item \textbf{Books}: {\em Processes in Microbial Ecology, Kirchman (2012)}
    \end{rSubsection}

%--Service----------------------------------------------------------------------
    \begin{rSubsection}{Committee Member}{}{}{}
        \item University of Illinois: {\em Research Scientist Search Committee}
        \item Indiana University: {\em IU Biology GRW Transportation}
        \item Michigan State University: {\em MMG Grad Committee}
    \end{rSubsection}

%--Service----------------------------------------------------------------------
    \begin{rSubsection}{Outreach}{}{}{}
        \item Jim Holland Summer Enrichment Program (Indiana University):
        {\em Stream Health \& Micro-Invertebrate\\Assessment} (2012 -- 2016)
        \item Wonderlab Science Museum (Bloomington, IN): {\em Good Microbes
        Exhibit} (2012 -- 2014)
        \item Women in STEM (Indiana University): {\em Stream Micro-Invertebrate
        \\ Assessment} (2012)
    \end{rSubsection}

%--Service----------------------------------------------------------------------
    \begin{rSubsection}{Professional Society Leadership}{}{}{}
        \item Secretary, Microbial Ecology Section, Ecological Society of America (2018 -- Present)
        \item Graduate Student Liaison for Microbial Ecology Section, ESA
              (2013 -- 2015)
    \end{rSubsection}

%--Service----------------------------------------------------------------------
    %\begin{rSubsection}{Special Session Co-Organizer}{}{}{}
    %    \item The Dance Between Microbes and Molecules
		%		\item ASLO Summer Meeting, Santa Fe NM (2016)
    %\end{rSubsection}

%--Service----------------------------------------------------------------------
    \begin{rSubsection}{Symposium Moderator}{}{}{}
        \item Turn and Face the Strain: Changing Signatures of Niche
        Processes in Disease and Community Diversity
			\item Ecological Society of America Annual Meeting. Portland, OR (2017)
    \end{rSubsection}

%--Service----------------------------------------------------------------------
    \begin{rSubsection}{Conference Co-Organizer}{}{}{}
        \item 2015 Midwest Ecology and Evolution Conference.
				Bloomington, IN
    \end{rSubsection}

%--Service----------------------------------------------------------------------
    \begin{rSubsection}{Course Development}{}{}{}
        \item Evolution of Molecules and Cells (2018, University of Illinois)
        \item Quantitative Biodiversity (2015, Indiana University)
        \item \hspace{2ex} repo: https://github.com/quantitativebiodiversity
        \item \hspace{2ex} website:
        http://documentup.com/QuantitativeBiodiversity/QuantitativeBiodiversity
    \end{rSubsection}

\end{rSection}

%-------------------------------------------------------------------------------
%	PROFESSIONAL AFFILIATIONS SECTION
%-------------------------------------------------------------------------------

\begin{rSection}{Professional Affiliations}
  \begin{tabular}{ @{} >{\em}l @{\hspace{6ex}} l }
    Ecological Society of America &  \\
    Association for the Sciences of Limnology and Oceanography &  \\
    Great Lakes Ecological Observatory Network &  \\
    %American Society for Microbiology & \\
  \end{tabular}

\end{rSection}

\pagebreak

%-------------------------------------------------------------------------------
%	INVITED SEMINARS
%-------------------------------------------------------------------------------

\begin{rSection}{Invited Seminars}

%--Presentation-----------------------------------------------------------------
  {\bf 2019} Department of Ecology \& Evolutionary Biology, University of Arizona, AZ, USA

%--Presentation-----------------------------------------------------------------
  {\bf 2019} D\'epartement des Sciences Biologiques, Universit\'e du Qu\'ebec \`a Montr\'eal, QC, CA
  
  %--Presentation-----------------------------------------------------------------
  {\bf 2019} Department of Biology \& Wildlife, University of Alaska Fairbanks, AK, USA

%--Presentation-----------------------------------------------------------------
  {\bf 2018} Gosnell School of Life Sciences, Rochester Institute of Technology, NY, USA

%--Presentation-----------------------------------------------------------------
  % {\bf Muscarella ME} (2018) Microbes and Molecules: How resources and traits shape communities. \\ Division of Biology, Kansas State University
  {\bf 2018} Division of Biology, Kansas State University, KS, USA

%--Presentation-----------------------------------------------------------------
  % {\bf Muscarella ME} (2018) Microbes and Molecules: How resources and traits shape communities. \\ Department of Ecology \& Evolutionary Biology, University of Kansas.
  {\bf 2018}  Department of Ecology \& Evolutionary Biology, University of Kansas, KS, USA

%--Presentation-----------------------------------------------------------------
 % {\bf Muscarella ME} (2015) Microbes and resources: individuals, communities, and ecosystem function. \\Program in Ecology, Evolution, and Conservation Biology, University of Illinois at Urbana-Champaign.
  {\bf 2015} Program in Ecology, Evolution, and Conservation Biology, University of Illinois, IL, USA

\end{rSection}

%-------------------------------------------------------------------------------
%	INVITED LECTURES
%-------------------------------------------------------------------------------

\begin{rSection}{Invited Lectures}

%--Presentation-----------------------------------------------------------------
  {\bf Muscarella ME} (2017) Stochastic Models in Evolutionary Ecology.
	Theoretical Biology \& Models (IB 494), University of Illinois at Urbana-Champaign.

\end{rSection}

%-------------------------------------------------------------------------------
%	CONTRIBUTED PRESENTATIONS
%-------------------------------------------------------------------------------

\begin{rSection}{Contributed Presentations}

%--Presentation-----------------------------------------------------------------
  Peralta AL, Stucy A, {\bf Muscarella ME} (2018) 
  Bacterial phylogeny-function relationships in freshwater ecosystems are particularly vulnerable to freshwater-brackish water intermediate environments.
  Annual Meeting of the North Carolina American Society for Microbiology Branch. Asheville, NC.

%--Presentation-----------------------------------------------------------------
  Wisnoski NI, {\bf Muscarella ME}, Lennon JT (2018) Dispersal and dormancy
  across ecosystem boundaries: Bacterial diversity and function along a
  reservoir transect. Association for the Sciences of Limnology and
  Oceanography. Victoria, British Columbia.

%--Presentation-----------------------------------------------------------------
  {\bf Muscarella ME}, O'Dwyer JP (2017) A Phylogenetic Framework for Trait
   Innovation and Selection in Microbial Communities.
   Ecological Society of America Annual Meeting, Portland OR.

%--Presentation-----------------------------------------------------------------
  Peralta AL, {\bf Muscarella ME} (2017) How can we manage microbial functions
  to restore ecosystem services in human-dominated landscapes?
  Ecological Society of America Annual Meeting, Portland OR.

%--Presentation-----------------------------------------------------------------
  Locey KJ, Lennon JT, Larsen ML, {\bf Muscarella ME}, Jones SE (2017)
  Spatiotemporal effects of microbial seed banks on community structure.
  Ecological Society of America Annual Meeting, Portland OR.

%--Presentation-----------------------------------------------------------------
  {\bf Muscarella ME}, Lennon JT (2015) Bacterial growth efficiency: do
  consumer and resource diversity influence the fate of carbon in aquatic
  ecosystems? Ecological Society of America 100th Annual Meeting, Baltimore MD.

%--Presentation-----------------------------------------------------------------
  {\bf Muscarella ME}, Locey KJ, Nevo E, Raz S, Lennon JT (2014) Microbial
  community assembly at Evolution Canyon: Does dormancy dilute the effects of
  dispersal and filtering? Ecological Society of America 99th Annual Meeting,
  Sacramento CA.

%--Presentation-----------------------------------------------------------------
  {\bf Muscarella ME}, Bird KC, Larsen ML, Placella SA, Lennon JT (2014)
  Phosphorus resource heterogeneity affects the structure and function of
  microbial food webs. Association for the Sciences of Limnology and
  Oceanography, Portland OR.

%--Presentation-----------------------------------------------------------------
  {\bf Muscarella ME}, Bird KC, Larsen ML, Placella SA, Lennon JT (2014)
  Phosphorus resource heterogeneity affects the structure and function of
  microbial food webs. Midwest Ecology and Evolution Conference, Dayton OH.

%--Presentation-----------------------------------------------------------------
  {\bf Muscarella ME}, Jones SE, Lennon JT (2013) Species Sorting Along a
  Subsidy Gradient Affects Community Stability. Ecological Society of America
  98th Annual Meeting, Minneapolis MN.

%--Presentation-----------------------------------------------------------------
  {\bf Muscarella ME}, Jones SE, Lennon JT (2013) Life in brown waters: Aquatic
  bacteria respond to increased terrestrial carbon loading. Midwest Ecology and
  Evolution Conference, South Bend, IN.

%--Presentation-----------------------------------------------------------------
  Lennon JT, {\bf Muscarella ME}, Jones SE (2013) Bacteria and browning:
  implications of terrestrial carbon subsidies for aquatic ecosystems.
  Association for the Sciences of Limnology and Oceanography, New Orleans, LA.

%--Presentation-----------------------------------------------------------------
  {\bf Muscarella ME}, Jones SE, Lennon JT (2013) Life in brown waters: Aquatic
  bacterial responses to increased terrestrial carbon loading. Association for
  the Sciences of Limnology and Oceanography, New Orleans, LA

%--Presentation-----------------------------------------------------------------
  {\bf Muscarella ME}, Jones SE, Lennon JT (2012) Life in brown waters: Aquatic
  bacteria respond to increased terrestrial carbon loading. LTER All Scientists
  Meeting, Estes Park, CO.

%--Presentation-----------------------------------------------------------------
  Booth MG, Gifford S, Doherty M, {\bf Muscarella M}, and Moran MA (2010)
  Expression of two carbon metabolism genes identified thorough transcriptomic
  analysis of estuarine bacterial communities of Sapelo Island, GA, U.S.A.
  International Society of Microbial Ecology Conference, Seattle, WA.

%--Presentation-----------------------------------------------------------------
  Doherty M, Poretsky R, {\bf Muscarella ME}, Moran MA, Booth MG (2009) Tracking
  Metabolism of an Important Terrestrial Carbon Source by Marine
  Bacterioplankton. Coastal and Estuarine Research Federation 20th Biennial
  Conference. Portland, OR.

%--Presentation-----------------------------------------------------------------
  {\bf Muscarella M}, Awong-Taylor J, Zettler J (2009) The Inability of
  Pathogenic and Nonpathogenic Strains of Escherichia coli to Survive in Coastal
  Beach Sand. 2009 Annual Conference of the Southeastern Branch American Society
  for Microbiology. Savannah, GA.

%--Presentation-----------------------------------------------------------------
  {\bf Muscarella M}, Awong-Taylor J (2008) Role of Curli Fibers in the Adhesion
  of \emph{Escherichia coli} O157:H7 to Bean Sprouts. American Society for Microbiology
  108th Annual Meeting. Boston, MA.

%--Presentation-----------------------------------------------------------------
  {\bf Muscarella M}, Awong-Taylor J, Zettler J (2008) The Potential for
  Pathogenic and non-pathogenic \emph{Escherichia coli} to Survive in Beach Sand from
  Tybee Island, GA. Association of Southeastern Biologists 69th Annual Meeting.
  Spartanburg, SC.

%--Presentation-----------------------------------------------------------------
  {\bf Muscarella M}, Awong-Taylor J (2008) A Method to Compare the
  Effectiveness of Different Washes on Removal Rates of \emph{E. coli} from Plant
  Roots. Southern Regional Honors Council 36th Annual Conference. Birmingham,
  Al.

%--Presentation-----------------------------------------------------------------
  Griffiths L, Bass  C, {\bf Muscarella M}, Awong-Taylor J, Craven K (2006)
  Using Traditional and Molecular Techniques to Determine if Microbial
  Contamination Plays a Role in Embryonic Death of Loggerhead Sea Turtle Eggs.
  Florida Academy of Sciences 70th Annual Meeting. Melbourne, Fl

\end{rSection}

%-------------------------------------------------------------------------------
%	ACADEMIC ADVISORS
%-------------------------------------------------------------------------------

\begin{rSection}{Academic Advisors}
  {\bf James P. O'Dwyer} University of Illinois (Postdoc)

  {\bf Jay T. Lennon}, Indiana University (Ph.D.)

  {\bf Judy Awong-Taylor}, Armstrong State University (B.S.)

\end{rSection}

%-------------------------------------------------------------------------------
%	MENTEES
%-------------------------------------------------------------------------------

\begin{rSection}{Mentees}

 % {\bf \em Teaching Assistants}: \\
    %Charles Dean ({\em University of Illinois} 2018)

  {\bf \em Undergraduate Students}: \\
    Belen Muniz ({\em University of Illinois} 2017)\\
    Mollie Carrison ({\em Indiana University} 2015 -- 2016)\\
    Rachel Ferrill ({\em Transylvania University} 2015) \\
    Xia Meng Howey ({\em Indiana University} 2012 -- 2014)

  {\bf \em High School Students}: \\
    Dakayla Calhoun ({\em Shortridge International Baccalaureate High
    School, Indianapolis, IN} 2015)\\
    Nick Nelson ({\em Bishop Noll Institute, Hammond, IN} 2012)

\end{rSection}


\end{document}
